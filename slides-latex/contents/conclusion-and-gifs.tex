\section{Conclusão}
\begin{frame}[fragile]
  \frametitle{GIFs}
  \begin{center}
    \animategraphics[autoplay, loop, width=0.8\linewidth]{2}{stick_}{1}{6} % 2 is the fps. It makes a gif with images stick_i where i in {1,...,6}
  \end{center}
\end{frame}

\begin{frame}[fragile]
  \frametitle{GIFs}
  \begin{center}
    \animategraphics[autoplay, loop, width=0.8\linewidth]{2}{flag1_}{1}{6}
  \end{center}
\end{frame}

\begin{frame}[fragile]
  \frametitle{GIFs}
  \begin{center}
    \animategraphics[autoplay, loop, width=0.8\linewidth]{2}{flag2_}{1}{6}
  \end{center}
\end{frame}

\begin{frame}[fragile]
  \frametitle{GIFs}
  \begin{center}
    \animategraphics[autoplay, loop, width=0.8\linewidth]{2}{molecule_}{1}{8}
  \end{center}
\end{frame}

\begin{frame}[fragile]
  \frametitle{GIFs}
  \begin{center}
    \animategraphics[autoplay, loop, width=0.8\linewidth]{2}{noisy_}{1}{6}
  \end{center}
\end{frame}

\begin{frame}[fragile]
  \frametitle{GIFs}
  \begin{center}
    \animategraphics[autoplay, loop, width=0.8\linewidth]{2}{translated_}{1}{6}
  \end{center}
\end{frame}

\begin{frame}
  \begin{columns}
    \column{0.48\textwidth}
    \textbf{Pros}
    \begin{itemize}
      \item Erro garantidamente não aumenta;
      \item Útil quando há um bom chute inicial;
      \item Consegue lidar com dados a mais, que não fazem parte do padrão procurado;
      \item Versátil: se sua noção de padrão não for só ortogonal, você pode trocar a minimização pra atacar problemas distintos.
    \end{itemize}
    \column{0.48\textwidth}
    \textbf{Cons}
    \begin{itemize}
      \item Heurística (auto-explicativo);
      \item Altamente dependente do ponto inicial;
      \item Um SVD a cada passo do algoritmo;
    \end{itemize}
  \end{columns}
\end{frame}

