
\section{Problema Desacoplado}

\begin{frame}[fragile]
  \frametitle{Problema Desacoplado}

  \begin{center}
    Se o problema fosse \[\argmin{Q} \fnorm{QA .+ \tau e\t - B},\] nós saberíamos resolvê-lo centralizando os pontos e usando o SVD!

  \pause

  \[\argmin{Q} \fnorm{QA - B} = U V\t,\]

  onde $B A\t = U \E V\t$.

  \pause

  \[\argmin{Q} \fnorm{QA .+ \tau e\t - B} = \bar{U} \bar{V}\t,\]

  onde $\bar{B} \bar{A}\t = \bar{U} \bar{\E} \bar{V}\t$ e $\bar{B}$ e $\bar{A}$ são $B$ e $A$ centralizados.
  \end{center}
\end{frame}

\begin{frame}[fragile]
  \frametitle{Problema Desacoplado}
  \begin{code}
    function procrustes(X, Y)
      U, _, V = svd(Y * X')
      return U * V'
    end
  \end{code}
  \pause
  \begin{code}
    function best_rigid_transf(X, Y)
      _, n = size(X)

      x = 1/n * X * ones(n)
      y = 1/n * Y * ones(n)

      Q = procrustes(X .- x, Y .- y)

      return Q, y - Q * x
    end
  \end{code}
\end{frame}



\begin{frame}[fragile]
  \frametitle{Problema Desacoplado}

  \begin{center}
    Se o problema fosse \[\argmin{P} \fnorm{A - BP},\] saberíamos resolvê-lo usando \texttt{for}!

    \pause
    \vspace{1cm}
    Verificando todas as distâncias das colunas de uma matriz para a outra e colocando o valor $1$ na coluna da matriz $P$ relativa ao ponto de menor distância.
  \end{center}
\end{frame}

\begin{frame}[fragile]
  \frametitle{Problema Desacoplado}
  \begin{code}
    function best_indicator(X, Y)
      nx, ny = size(X, 2), size(Y, 2)
      P = zeros(ny, nx)

      for i in 1:nx
        best_j = argmin([norm(X[:, i] - Y[:, j]) for j in 1:ny])
        P[best_j, i] = 1
      end

      return P
    end
  \end{code}
\end{frame}
