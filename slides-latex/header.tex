
%->8-------------------------------------------------------------------------------------------8<-%
\usepackage[brazil]{babel}
\usepackage[utf8]{inputenc}
\usepackage[T1]{fontenc}
%->8-------------------------------------------------------------------------------------------8<-%

%->8-------------------------------------------------------------------------------------------8<-%
\usepackage{graphicx}
\usepackage{float}
\usepackage{geometry}
\usepackage{csquotes}
\usepackage{upquote}
\usepackage{hyperref}
\usepackage{xcolor}
\usepackage{animate}
\geometry{left=0.4cm, right=0.4cm}
\newcommand{\red}[1]{\textcolor{red}{#1}}
\newcommand{\blue}[1]{\textcolor{blue}{#1}}
%->8-------------------------------------------------------------------------------------------8<-%

%->8-------------------------------------------------------------------------------------------8<-%
\usepackage{amsmath, amssymb, mathdots}
\newcommand{\fnorm}[1]{\left\lVert#1\right\rVert_\mathsf{F}}
\newcommand{\norm}[1]{\left\lVert#1\right\rVert}
\newcommand{\argmintext}{\text{ArgMin}}
\newcommand{\argmin}[1]{\underset{#1}{\argmintext}\,\,}
\renewcommand{\t}{^{\mathsf{T}\mkern-1mu}}
\newcommand{\E}{\Sigma}
\newcommand{\bigvert}{\rule[-1ex]{0.5pt}{6.8ex}}
\renewcommand{\vert}{\rule[-1ex]{0.5pt}{2.5ex}}
%->8-------------------------------------------------------------------------------------------8<-%

%->8-------------------------------------------------------------------------------------------8<-%
\usepackage[style=numeric, backend=biber]{biblatex}
\addbibresource{repertoire.bib}
\graphicspath{{figures/}}
%->8-------------------------------------------------------------------------------------------8<-%

%->8-------------------------------------------------------------------------------------------8<-%
\usepackage{listings}
\lstnewenvironment{code}[1][]{
  \lstset{
    basicstyle=\ttfamily,
    columns=flexible,
    breaklines=true,
    breakatwhitespace=true,
    frame=none,
    basewidth=0.5em,
    aboveskip=13pt,
    belowskip=0pt,
    #1
  }
}{}

\newcommand{\excerptpage}[2]{
  \begin{center}
    \textit{#1}\\
        --- #2.
  \end{center}
}

\newcommand{\var}[1]{\texttt{#1}}
%->8-------------------------------------------------------------------------------------------8<-%

%->8-------------------------------------------------------------------------------------------8<-%
\usetheme{Darmstadt} % Blue and white theme that John always uses
% \usetheme{Warsaw}
% \setbeamercolor{normal text}{fg=white,bg=black!90}
% \setbeamercolor{structure}{fg=white}
% \setbeamercolor{alerted text}{fg=red!85!black}
% \setbeamercolor{item projected}{use=item,fg=black,bg=item.fg!35}
% \setbeamercolor*{palette primary}{use=structure,fg=structure.fg}
% \setbeamercolor*{palette secondary}{use=structure,fg=structure.fg!95!black}
% \setbeamercolor*{palette tertiary}{use=structure,fg=structure.fg!90!black}
% \setbeamercolor*{palette quaternary}{use=structure,fg=structure.fg!95!black,bg=black!80}
% \setbeamercolor*{framesubtitle}{fg=white}
% \setbeamercolor*{block title}{parent=structure,bg=black!60}
% \setbeamercolor*{block body}{fg=black,bg=black!10}
% \setbeamercolor*{block title alerted}{parent=alerted text,bg=black!15}
% \setbeamercolor*{block title example}{parent=example text,bg=black!15}
% \setbeamertemplate{headline}{}
\setbeamertemplate{navigation symbols}{}
\setbeamertemplate{headline}{
  \leavevmode
  \hbox{\begin{beamercolorbox}[wd=\paperwidth,ht=2.5ex,dp=1.125ex]{section in head/foot}
  \insertsectionnavigationhorizontal{\paperwidth}{}{}
  \end{beamercolorbox}}
}
%->8-------------------------------------------------------------------------------------------8<-%

%->8-------------------------------------------------------------------------------------------8<-%
\title{Iterative Closest Point}
\subtitle{Encontrando Padrões em Dados}
\author{Gustavo de Mendonça Freire \and João Victor Lopez Pereira}
\institute{Instituto de Computação -- UFRJ}
\date{2025}
%->8-------------------------------------------------------------------------------------------8<-%
